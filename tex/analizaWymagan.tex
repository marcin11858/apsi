\chapter{Analiza wymagań}

\section{Wstęp}

%TODO Alternatywna wersja 

Wraz z rozwojem handlu konieczne stało się stowrznie miejsca w którym towary
będą składowane przed zakupem  przez klienta. W przypadku małych magazynów to
pracownicy są w stanie nim efektywnie zarządzać i efektywnie wyszukiwać towarów
znajdujących się w nim. Jednak co dzieje się w przypadku gdy magazyn jest
większy? Konieczne staje się stowrzenie systemu który umożliwi pracownikom
łatwiejsze i bardziej efektywne zarządzanie towarami znajdującymi się w
magazynie.

Głównym zadaniem systemu do zarządzania magazynem jest przechowywanie ilości
poszczególnych towarów. Możliwość przyjmowania dostaw jaki i również sprzedaży
towarów klientom. Powinien również przechowywać informacje o dostawcach,
towarach a także o klientach.

\subsection{Przeznaczenie systemu}

Głównym zadaniem systemu jest ułatwienie zarządzania magazynem pracownikom
magazynu, poprzez umożliwienie zarządzania klientami, dostawcami, towarami a
także zamównieniam sprzedaży jak i zakupu. System ten nie jest specjalizowany
pod konkrentną dziedzine handlu. Umożliwia on natomiast przechowywanie i
zarządznanie informacjami o towarach dowolnego typu. 

\subsection{Architektura systemu}
% Świetny styl!
W obecnie towrzonych systemach spotyka się dwie podstawowe struktury:
architekturę klient-serwer oraz architekturę trójwarstwową.
W architekturze klient-serwer, która była szczególnie popularna w
latach dziewięćdziesiątych ubiegłego wieku, wyróżnia się
dwie warstwy: 
\begin{itemize}
 \item aplikację użytkownika (\emph{klient});
 \item system zarządzania bazą danych (\emph{serwer}).
\end{itemize}

Struktura ta sprawdza się dla prostych systemów, których zadaniem jest
zapisywanie, odczytywanie oraz aktualizacja danych. Problem
pojawia się jednak w przypadku, gdy dane muszą być przetwarzane w
nietrywialny sposób, przy uwzględnieniu dziedziny problemu (ang. \emph{domain
logic}) modelowanego zagadnienia. Realizacja obliczeń w warstwie klienta,
której głównym zadaniem jest prezentacja informacji użytkownikowi, może
znacząco wpływać na jego komfort pracy oraz powodować problemy związane z
duplikacją kodu źródłowego aplikacji.
Natomiast umieszczenie logiki aplikacji po stronie serwera bardzo często narzuca
przygotowanie programu w środowisku specyficznym dla danego systemu zarządzania
bazami danych. 

Problemy te zmusiły projektantów aplikacji do wydzielenia jeszcze jednego
poziomu, który jest odpowiedzialny za logikę operacji na danych. W
architekturze trójwarstwowej uwzględnione są następujące warstwy:
\begin{itemize}
 \item warstwa prezentacji;
 \item warstwa aplikacji;
 \item warstwa źródła danych.
\end{itemize}

%TODO Wstawić schemat prezentujący architekturę trójwarstwową.

W przypadku systemu do obsługi magazynu zdecydowano się wykorzystanie
architektóry trójwarstwowej.

\section{Aktorzy}

Aktorzy w zaprojektowanym systemie zostali podzieleni na dwie grupy: aktorów
osobowych oraz aktorów nieosobowych. Do aktorów osobowym można zaliczyć
wszystkich użytkowników projektowanego systemu, aktorami nieosobowymi są
natomiast wszystkie systemy zewnętrzne współpracujące z projektowanym systemem.
Poniższy diagram przedstawia hierarchie aktorów osobowych w systemie:

%TODO Wstawić diagram z aktorami osobowymi: administrator, magazynier,
% sprzedawca 

Diagram poniżej prezentuje natomiast aktorów nieosobowych:

%TODO Wstawić diagram z systemami zewnętrznymi: system do fakturowaniam system
% do zamówień dostawców.


\section{Wymagania funkcjonalne}

Niniejsze rozdział zawiera zdefiniowane dla systemu wymagania funkcjonalne.

%\arrayrulecolor{line}
%\rowcolors{2}{cell}{white}

\begin{table}[ht]
	 \begin{center}
% 	    \rowcolors{1}{}{lightblue}
% TODO a wyszukiwanie
	    \begin{tabular}{| l | l | l | l | l |}%\toprule
	    	\hline
		    Lp. & Nazwa  & Priorytet & Ryzyko & Nazwa aktora \\
		    \hline
		    1 & Zarządzanie użytkownikami & wysoki & niskie & administrator \\
		    1.1 & Dodawanie użytkowników & wysoki & niskie & adminstrator \\
		    1.2 & Edycja danych użytkowników & wysoki & niskie & adminstrator \\ 	
		    1.3 & Usuwanie danych użytkowników & wysoki &niskie & administrator \\
		    \hline
		    2 & Zarządzanie danymi towarów & wysoki & niskie & magazynier \\
		    2.1 & Dodawanie towaru & wysoki &  niskie & magazynier \\
		    2.2 & Edycja danych twarów & wysoki & niskie & magazynier \\
		    2.3 & Usuwanie danych towarów & wysoki & niskie & magazynier \\
		    \hline
		   	3 & Zarządzanie danymi klientów & wysoki & niskie & sprzedawca \\
		   	3.1 & Dodawanie klientów & wysoki & niski & sprzedawca \\
		   	3.2 & Edycja danych klientów & wysoki & niskie & sprzedawca \\
		   	3.3 & Usuwanie danych klientów & wysoki & niskie & sprzedawca \\
		   	\hline
		   	4 & Zarządzanie danycmi dostawców & wysoki & niskie & magazynier \\
		   	4.1 & Dodawanie dostawców & wysoki & niskie & magazynier \\
		   	4.2 & Edycja danych dostawców & wysoki & niskie & magaziner \\
		   	4.3 & Usuwanie danych dostawców & wysoki & niskie & magazynier \\
		   	\hline
		   	5 & Zarządzanie zamowieniami sprzedaży & wysoki & niskie & sprzedawca \\
		   	5.1 & Tworzenie zamówień sprzedaży & wyskoki & niskie & sprzedawca \\
		   	5.2 & Edycja zamówień sprzedaży & wysoki & niskie & sprzedaca \\
		   	5.3 & Usuwanie zamówień sprzedaży & wysoki & niskie & sprzedaca \\
		   	5.4 & Realizacja zamówień sprzedaży & wysoki & niskie & sprzedawca \\
		   	5.5 & Anulowanie zamówień sprzedaży & wysoki & niskie & sprzedawca \\
		   	\hline
		   	6 & Zarządzanie zamówieniami zakupu & wysoki & niskie & magazynier \\
		   	6.1 & Tworzenie zamowień zakupu & wysoki & niskie & magazynier \\
		   	6.2 & Edycja zamówień zakupu & wysoki & niskie & magazynier \\
		   	6.3 & Usuwanie zamówień zakupu & wysoki & niskie & magazynier \\
		   	6.4 & Realizacja zamówień zakupu & wysoki & niskie & magazynier \\
		   	6.5 & Anulowanie zamówień zakupu & wysoki & niskie & magazynier \\
		   	\hline
		   	7 & Zarządzanie raportami & średni & niskie & administrator \\
		   	7.1 & Generowanie raportów sprzedaży & średni & niskie & administrator \\
		   	7.2 & Generowanie raportów zakupów & średni & niskie & sdministrator \\
		   	\hline
		   	
		   	\hline
	    \end{tabular}
	\end{center}
\end{table}
\FloatBarrier

\section{Wymagania niefunkcjonalne}

\section{Specyfikacja przypadków użycia na poziomie ogólnym}