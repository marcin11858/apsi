\chapter{Analiza wymagań}

\section{Wstęp} 

Wraz z rozwojem handlu konieczne stało się stowrznie miejsca w którym towary
będą składowane przed zakupem  przez klienta. W przypadku małych magazynów to
pracownicy są w stanie nim efektywnie zarządzać i efektywnie wyszukiwać towarów
znajdujących się w nim. Jednak co dzieje się w przypadku gdy magazyn jest
większy? Konieczne staje się stowrzenie systemu który umożliwi pracownikom
łatwiejsze i bardziej efektywne zarządzanie towarami znajdującymi się w
magazynie.

Głównym zadaniem systemu do zarządzania magazynem jest przechowywanie ilości
poszczególnych towarów. Możliwość przyjmowania dostaw jaki i również sprzedaży
towarów klientom. Powinien również przechowywać informacje o dostawcach,
towarach a także o klientach.

\subsection{Przeznaczenie systemu}

Głównym zadaniem systemu jest ułatwienie zarządzania magazynem pracownikom
magazynu, poprzez umożliwienie zarządzania klientami, dostawcami, towarami a
także zamównieniam sprzedaży jak i zakupu. System ten nie jest specjalizowany
pod konkrentną dziedzine handlu. Umożliwia on natomiast przechowywanie i
zarządznanie informacjami o towarach dowolnego typu. 



\section{Aktorzy}

\section{Wymagania funkcjonalne}


\section{Wymagania niefunkcjonalne}

\section{Specyfikacja przypadków użycia na poziomie ogólnym}