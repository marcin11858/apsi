\chapter{Analiza wymagań}

\section{Wstęp} 

Wraz z rozwojem handlu konieczne stało się stowrznie miejsca w którym towary
będą składowane przed zakupem  przez klienta. W przypadku małych magazynów to
pracownicy są w stanie nim efektywnie zarządzać i efektywnie wyszukiwać towarów
znajdujących się w nim. Jednak co dzieje się w przypadku gdy magazyn jest
większy? Konieczne staje się stowrzenie systemu który umożliwi pracownikom
łatwiejsze i bardziej efektywne zarządzanie towarami znajdującymi się w
magazynie.

Głównym zadaniem systemu do zarządzania magazynem jest przechowywanie ilości
poszczególnych towarów. Możliwość przyjmowania dostaw jaki i również sprzedaży
towarów klientom. Powinien również przechowywać informacje o dostawcach,
towarach a także o klientach.

\subsection{Przeznaczenie systemu}

Głównym zadaniem systemu jest ułatwienie zarządzania magazynem pracownikom
magazynu, poprzez umożliwienie zarządzania klientami, dostawcami, towarami a
także zamównieniam sprzedaży jak i zakupu. System ten nie jest specjalizowany
pod konkrentną dziedzine handlu. Umożliwia on natomiast przechowywanie i
zarządznanie informacjami o towarach dowolnego typu. 

\subsection{Architektura systemu}

W obecnie towrzonych systemach spotyka się dwie podstawowe struktury:
architekturę klient-serwer oraz architekturę trójwarstwową.
W architekturze klient-serwer, która była szczególnie popularna w
latach dziewięćdziesiątych ubiegłego wieku, wyróżnia się
dwie warstwy: 
\begin{itemize}
 \item aplikację użytkownika (\emph{klient});
 \item system zarządzania bazą danych (\emph{serwer}).
\end{itemize}

Struktura ta sprawdza się dla prostych systemów, których zadaniem jest
zapisywanie, odczytywanie oraz aktualizacja danych. Problem
pojawia się jednak w przypadku, gdy dane muszą być przetwarzane w
nietrywialny sposób, przy uwzględnieniu dziedziny problemu (ang. \emph{domain
logic}) modelowanego zagadnienia. Realizacja obliczeń w warstwie klienta,
której głównym zadaniem jest prezentacja informacji użytkownikowi, może
znacząco wpływać na jego komfort pracy oraz powodować problemy związane z
duplikacją kodu źródłowego aplikacji.
Natomiast umieszczenie logiki aplikacji po stronie serwera bardzo często narzuca
przygotowanie programu w środowisku specyficznym dla danego systemu zarządzania
bazami danych. 

Problemy te zmusiły projektantów aplikacji do wydzielenia jeszcze jednego
poziomu, który jest odpowiedzialny za logikę operacji na danych. W
architekturze trójwarstwowej uwzględnione są następujące warstwy:
\begin{itemize}
 \item warstwa prezentacji;
 \item warstwa aplikacji;
 \item warstwa źródła danych.
\end{itemize}

%TODO Wstawić schemat prezentujący architekturę trójwarstwową.

W przypadku systemu do obsługi magazynu zdecydowano się wykorzystanie
architektóry trójwarstwowej.

\section{Aktorzy}

Aktorzy w zaprojektowanym systemie zostali podzieleni na dwie grupy: aktorów
osobowych oraz aktorów nieosobowych. Do aktorów osobowym można zaliczyć
wszystkich użytkowników projektowanego systemu, aktorami nieosobowymi są
natomiast wszystkie systemy zewnętrzne współpracujące z projektowanym systemem.
Poniższy diagram przedstawia hierarchie aktorów osobowych w systemie:

%TODO Wstawić diagram z aktorami osobowymi: administrator, magazynier,
% sprzedawca 

Diagram poniżej prezentuje natomiast aktorów nieosobowych:

%TODO Wstawić diagram z systemami zewnętrznymi: system do fakturowaniam system
% do zamówień dostawców.


\section{Wymagania funkcjonalne}


\section{Wymagania niefunkcjonalne}

\section{Specyfikacja przypadków użycia na poziomie ogólnym}