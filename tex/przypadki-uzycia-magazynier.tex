\section{Moduł zarządzania towarami}
% 
\singlespacing
\subsection{Zarządzanie danymi towarów}

Dane towarów stanowią podstawę do przygotowywania zamówień przez
sprzedawcę. Istotne jest więc, aby magazynier, odpowiedzialny za
zarządzanie towarami przedsiębiorstwa, posiadał narzędzia 
ułatwiające realizację niezbędnych czynności. System
do zarządzania magazynem powinien więc umożliwiać \textbf{zarządzanie
danymi towarów}.

\begin{usecase}
  % Pola na podstawie przykładowego szablonu z
  % http://piotrsalata.pl/documents/wymagania.pdf.  
  %
  % ID + nazwa
  \addtitle{PU30}{Zarządzanie danymi towarów}
  %
  % Priorytet
  \addfield{Priorytet:}{wysoki}
  %
  % Aktor główny - zazwyczaj ten, który inicjuje przypadek użycia.
  \addfield{Aktor główny:}{Magazynier}
  %
  %	Warunki początkowe
  \addfield{Warunki początkowe:}{Magazynier został uwierzytelniony.}
  %
  % Warunki końcowe (brak).
  % 
  % Scenariusz główny.
  \addscenario{Przebieg zdarzeń:}{
  \item Magazynier wybiera opcję wyświetlenia listy towarów w magazynie.
  \item System wyświetla listę towarów w magazynie.}
  %
  % Wyjątki \addfield{Wyjątki:}{brak} 
  %
  % Wymagania funkcjonalne związane z danym przypadkiem użycia
  \addfield{Wymagania funkcjonalne:}{2. Zarządzanie danymi towarów}
  %
  % Wymagania niefunkcjonalne - tylko gdy konieczne
  % \addfield{Wymagania niefunkcjonalne}{}
\end{usecase}
TODO diagram komunikacji
TODO diagram ecb.
\subsection{Dodawanie towarów}
Przypadek użycia systemu umożliwiający wprowadzenie do systemu informacji
o nowym towarze.
itd. tak jak w poprzednim przypadku użycia.
