\section{Moduł zarządzania towarami}
% 
\singlespacing
\subsection{Zarządzanie danymi towarów}

\begin{usecase}
  \addtitle{PU30}{Zarządzanie danymi towarów}
  \addfield{Priorytet:}{wysoki}
  \addfield{Aktor główny:}{Magazynier}
  \addfield{Warunki początkowe:}{Aktor został uwierzytelniony.}
  % Warunki końcowe (brak).
  \addscenario{Przebieg zdarzeń:}{
  \item Aktor wybiera opcję przeglądania listy towarów w magazynie.
  \item System wyświetla listę towarów w magazynie.
  }
  \addfield{Wymagania funkcjonalne:}{2.}
\end{usecase}
TODO diagram komunikacji
TODO diagram ecb.

%%%%%%%%
\subsection{Dodawanie towaru}
%
\begin{usecase}
  \addtitle{PU31}{Dodawanie towaru}
  \addfield{Priorytet:}{wysoki}
  \addfield{Aktor główny:}{Magazynier}
  \addfield{Rozszerza przypadki:}{PU30}
  \addfield{Warunki początkowe:}{Aktor został uwierzytelniony.}
  \additemizedfield{Warunki końcowe:}{ 
    \item Dane towaru zostały zapisane w systemie.
    \item Aktor otrzymuje informacje o poprawnym dodaniu towaru do systemu.
    \item Użytkownik może wyświetlić dane towaru na liście towarów. % TODO jak warunki końcowe mają się do scenariuszy alternatywnych?
    \item Dane towaru mogą być uwzględnione w dokumentach przedstawionych w rozdziale \ref{dziedzina-problemu}.
  }
  \addscenario{Przebieg zdarzeń:}{
    \item Aktor wybiera opcję dodania nowego towaru do systemu.
    \item System wyświetla formularz dodawania nowego towaru do systemu.
    \item Aktor wpisuje wymagane oraz opcjonalne dane do formularza.
    \item Aktor wybiera opcję zapisania nowego towaru w systemie.
    \item System informuje aktora, że towar został poprawnie zapisany w systemie.
  }
  \addscenario{Przebiegi alternatywne:} {
    \item [4.a] Aktor nie podał wymaganych pól formularza:
      \begin{enumerate}
        \item[1--4.] Jak w scenariuszu głównym.
        \item[5.] System wyświetla powiadomienie o konieczności podania wymaganych informacji.
        \item[6.] Aktor wraca do punktu 3.  
      \end{enumerate}
    \item [4.b] Aktor podał błędne wartości pól formularza:
      \begin{enumerate}
        \item[1--4.] Jak w scenariuszu głównym.
        \item[5.] System wyświetla powiadomienie o błędnych polach formularza.
        \item[6.] Aktor wraca do punktu 3.
      \end{enumerate}
  }
  \addfield{Zakres przetwarzanych danych:} {
    Dane towaru takie jak przedstawione w rozdziale \ref{dziedzina-problemu}.
  }
  \addfield{Warunki poprawności danych:}{
    Warunki poprawności takie jak przedstawione w rozdziale \ref{dziedzina-problemu}.
  }
  \addfield{Wymagania funkcjonalne:}{2.1}
\end{usecase}
TODO diagram komunikacji
TODO diagram ecb.

%%%%%%%
\subsection{Edycja opisu towaru}
\begin{usecase}
  \addtitle{PU32}{Edycja opisu towaru}
  \addfield{Priorytet:}{wysoki}
  \addfield{Aktor główny:}{Magazynier}
  \addfield{Rozszerza przypadki:}{PU30}
  \addfield{Warunki początkowe:}{Aktor został uwierzytelniony.}
   \additemizedfield{Warunki końcowe:}{ 
    \item Dane towaru zostały zaktualizowane w systemie.
    \item Aktor otrzymuje informację o poprawnej aktualizacji danych towaru.
    \item Użytkownik może wyświetlić dane towaru na liście towarów. 
    \item Dane towaru mogą być uwzględnione w dokumentach przedstawionych w rozdziale \ref{dziedzina-problemu}.
  }
  \addscenario{Przebieg zdarzeń:}{
    \item Aktor wybiera opcję aktualizacji danych wskazanego towaru.
    \item System wyświetla formularz aktualizacji danych towaru wypełniony obecnymi danymi towaru.
    \item Aktor wypełnia lub zmienia wybrane pola formularza.
    \item Aktor wybiera opcję aktualizacji danych towaru.
    \item System informuje aktora, że dane towaru zostały poprawnie zaktualizowane.
  }
  % TODO duplikacja przebiegów alternatywnych z PU31
  \addscenario{Przebiegi alternatywne:} {
    \item [4.a] Aktor nie podał wymaganych pól formularza:
      \begin{enumerate}
        \item[1.--4.] Jak w scenariuszu głównym.
        \item[5.] System wyświetla powiadomienie o konieczności podania wymaganych informacji.
        \item[6.] Aktor wraca do punktu 3.  
      \end{enumerate}
    \item [4.b] Aktor podał błędne wartości pól formularza:
      \begin{enumerate}
        \item[1.--4.] Jak w scenariuszu głównym.
        \item[5.] System wyświetla powiadomienie o błędnych polach formularza.
        \item[6.] Aktor wraca do punktu 3.
      \end{enumerate}
  }
  \addfield{Zakres przetwarzanych danych:}{Dane towaru dostępne przy edycji, przedstawione w rozdziale \ref{dziedzina-problemu}.}
  \addfield{Warunki poprawności danych:}{Warunki poprawności takie jak przedstawione w rozdziale \ref{dziedzina-problemu}.}
  \addfield{Wymagania funkcjonalne:}{2.2}
\end{usecase}
TODO diagram komunikacji TODO diagram ecb
%%%%%%%
\subsection{Usuwanie danych towaru}
\begin{usecase}
  \addtitle{PU35}{Usuwanie danych towaru}
  \addfield{Priorytet:}{wysoki}
  \addfield{Aktor główny:}{Magazynier}
  \addfield{Warunki początkowe:}{Aktor został uwierzytelniony.}
  \additemizedfield{Warunki końcowe:}{
    \item Dane towaru zostały usunięte z systemu.
    \item Usunięty towar nie jest wypisywany na liście towarów w magazynie.
  }
  \addscenario{Przebieg zdarzeń:}{
    \item Aktor wybiera opcję usunięcia danych towaru z systemu.
    \item System prosi o potwierdzenie operacji.
    \item Aktor potwierdza usunięcie towaru z systemu.
    \item System wyświetla informację o pomyślnym usunięciu towaru z systemu.
  }
  \addscenario{Przebiegi alternatywne:}{
    \item[4.a] Aktor anuluje usunięcie towaru z systemu.
      \begin{enumerate}
        \item[1.--2.] Jak w scenariuszu głównym.
        \item[3.] Aktor anuluje usunięcie towaru z systemu.
        \item[4.] System wyświetla informację, że operacja została anulowana.
      \end{enumerate}
  }
  \addfield{Wymagania funkcjonalne:}{2.3}
\end{usecase}
TODO diagram komunikacji TODO diagram ecb
% %%%%%%
% \subsection{Zarządzanie danymi dostawców}
% \begin{usecase}
%   %
%   % ID + nazwa
%   \addtitle{PU36}{Zarządzanie danymi dostawców}
%   %
%   % Priorytet
%   \addfield{Priorytet:}{wysoki}
%   %
%   % Aktor główny - zazwyczaj ten, który inicjuje przypadek użycia.
%   \addfield{Aktor główny:}{Magazynier}
%   %
%   %	Warunki początkowe
%   \addfield{Warunki początkowe:}{Magazynier został uwierzytelniony.}
%   %
%   % Warunki końcowe (brak).
%   % 
%   % Scenariusz główny.
%   \addscenario{Przebieg zdarzeń:}{
%   \item Magazynier wybiera opcję wyświetlenia listy towarów w
%     magazynie.
%   \item System wyświetla listę towarów w magazynie.
%   }
%   %
%   % Wyjątki \addfield{Wyjątki:}{brak}
%   %
%   % Wymagania funkcjonalne związane z danym przypadkiem użycia
%   \addfield{Wymagania funkcjonalne:}{4. Zarządzanie danymi dostawców}
%   %
%   % Wymagania niefunkcjonalne - tylko gdy konieczne
%   % \addfield{Wymagania niefunkcjonalne}{}
% \end{usecase}
% TODO diagram komunikacji TODO diagram ecb
% %%%%%%
% \subsection{Dodawanie dostawców}
% \begin{usecase}
%   %
%   % ID + nazwa
%   \addtitle{PU37}{Dodawanie dostawców}
%   %
%   % Priorytet
%   \addfield{Priorytet:}{wysoki}
%   %
%   % Aktor główny - zazwyczaj ten, który inicjuje przypadek użycia.
%   \addfield{Aktor główny:}{Magazynier}
%   %
%   %	Warunki początkowe
%   \addfield{Warunki początkowe:}{Magazynier został uwierzytelniony.}
%   %
%   % Warunki końcowe (brak).
%   % 
%   % Scenariusz główny.
%   \addscenario{Przebieg zdarzeń:}{
%   \item Magazynier wybiera opcję wyświetlenia listy towarów w
%     magazynie.
%   \item System wyświetla listę towarów w magazynie.
%   }
%   %
%   % Wyjątki \addfield{Wyjątki:}{brak}
%   %
%   % Wymagania funkcjonalne związane z danym przypadkiem użycia
%   \addfield{Wymagania funkcjonalne:}{4.1 Dodawanie dostawców.}
%   %
%   % Wymagania niefunkcjonalne - tylko gdy konieczne
%   % \addfield{Wymagania niefunkcjonalne}{}
% \end{usecase}
% TODO diagram komunikacji TODO diagram ecb
% %%%%%%
% \subsection{Edycja danych dostawców}
% \begin{usecase}
%   %
%   % ID + nazwa
%   \addtitle{PU38}{Edycja danych dostawców}
%   %
%   % Priorytet
%   \addfield{Priorytet:}{wysoki}
%   %
%   % Aktor główny - zazwyczaj ten, który inicjuje przypadek użycia.
%   \addfield{Aktor główny:}{Magazynier}
%   %
%   %	Warunki początkowe
%   \addfield{Warunki początkowe:}{Magazynier został uwierzytelniony.}
%   %
%   % Warunki końcowe 
%   \additemizedfield{Warunki końcowe:}{
%     \item dane dostawcy dla zamówień zakupu nie mogą być zmienione. (dane historyczne).
%   }
%   %  
%   % Scenariusz główny.
%   \addscenario{Przebieg zdarzeń:}{
%   \item Magazynier wybiera opcję wyświetlenia listy towarów w
%     magazynie.
%   \item System wyświetla listę towarów w magazynie.
%   }
%   %
%   % Wyjątki \addfield{Wyjątki:}{brak}
%   %
%   % Wymagania funkcjonalne związane z danym przypadkiem użycia
%   \addfield{Wymagania funkcjonalne:}{4.2 Edycja danych dostawców}
%   %
%   % Wymagania niefunkcjonalne - tylko gdy konieczne
%   % \addfield{Wymagania niefunkcjonalne}{}
% \end{usecase}
% TODO diagram komunikacji TODO diagram ecb
% %%%%%%%
% \subsection{Usuwanie danych dostawców}
% \begin{usecase}
%   %
%   % ID + nazwa
%   \addtitle{PU39}{Usuwanie danych dostawców}
%   %
%   % Priorytet
%   \addfield{Priorytet:}{wysoki}
%   %
%   % Aktor główny - zazwyczaj ten, który inicjuje przypadek użycia.
%   \addfield{Aktor główny:}{Magazynier}
%   %
%   %	Warunki początkowe
%   \addfield{Warunki początkowe:}{Magazynier został uwierzytelniony.}
%   %
%   % Warunki końcowe
%   \additemizedfield{Warunki końcowe:}{
%     \item Dane dostawcy dla zamówień zakupu muszą pozostać niezmienione.
%   }
%   % 
%   % Scenariusz główny.
%   \addscenario{Przebieg zdarzeń:}{
%   \item Magazynier wybiera opcję wyświetlenia listy towarów w
%     magazynie.
%   \item System wyświetla listę towarów w magazynie.
%   }
%   %
%   % Wyjątki \addfield{Wyjątki:}{brak}
%   %
%   % Wymagania funkcjonalne związane z danym przypadkiem użycia
%   \addfield{Wymagania funkcjonalne:}{Usuwanie danych towarów}
%   %
%   % Wymagania niefunkcjonalne - tylko gdy konieczne
%   % \addfield{Wymagania niefunkcjonalne}{}
% \end{usecase}
% TODO diagram komunikacji TODO diagram ecb


% \subsection{Zarządzanie zamówieniami zakupu}
% \begin{usecase}
%   %
%   % ID + nazwa
%   \addtitle{PU36}{Zarządzanie zamówieniami zakupu}
%   %
%   % Priorytet
%   \addfield{Priorytet:}{wysoki}
%   %
%   % Aktor główny - zazwyczaj ten, który inicjuje przypadek użycia.
%   \addfield{Aktor główny:}{Magazynier}
%   %
%   %	Warunki początkowe
%   \addfield{Warunki początkowe:}{Magazynier został uwierzytelniony.}
%   %
%   % Warunki końcowe (brak).
%   % 
%   % Scenariusz główny.
%   \addscenario{Przebieg zdarzeń:}{
%   \item Magazynier wybiera opcję wyświetlenia listy towarów w
%     magazynie.
%   \item System wyświetla listę towarów w magazynie.
%   }
%   %
%   % Wyjątki \addfield{Wyjątki:}{brak}
%   %
%   % Wymagania funkcjonalne związane z danym przypadkiem użycia
%   \addfield{Wymagania funkcjonalne:}{Usuwanie danych towarów}
%   %
%   % Wymagania niefunkcjonalne - tylko gdy konieczne
%   % \addfield{Wymagania niefunkcjonalne}{}
% \end{usecase}
% TODO diagram komunikacji TODO diagram ecb

% \subsection{Zarządzanie zamówieniami zakupu}
% \begin{usecase}
%   %
%   % ID + nazwa
%   \addtitle{PU36}{Zarządzanie zamówieniami zakupu}
%   %
%   % Priorytet
%   \addfield{Priorytet:}{wysoki}
%   %
%   % Aktor główny - zazwyczaj ten, który inicjuje przypadek użycia.
%   \addfield{Aktor główny:}{Magazynier}
%   %
%   %	Warunki początkowe
%   \addfield{Warunki początkowe:}{Magazynier został uwierzytelniony.}
%   %
%   % Warunki końcowe (brak).
%   % 
%   % Scenariusz główny.
%   \addscenario{Przebieg zdarzeń:}{
%   \item Magazynier wybiera opcję wyświetlenia listy towarów w
%     magazynie.
%   \item System wyświetla listę towarów w magazynie.
%   }
%   %
%   % Wyjątki \addfield{Wyjątki:}{brak}
%   %
%   % Wymagania funkcjonalne związane z danym przypadkiem użycia
%   \addfield{Wymagania funkcjonalne:}{Usuwanie danych towarów}
%   %
%   % Wymagania niefunkcjonalne - tylko gdy konieczne
%   % \addfield{Wymagania niefunkcjonalne}{}
% \end{usecase}
% TODO diagram komunikacji TODO diagram ecb

% \subsection{Zarządzanie zamówieniami zakupu}
% \begin{usecase}
%   %
%   % ID + nazwa
%   \addtitle{PU36}{Zarządzanie zamówieniami zakupu}
%   %
%   % Priorytet
%   \addfield{Priorytet:}{wysoki}
%   %
%   % Aktor główny - zazwyczaj ten, który inicjuje przypadek użycia.
%   \addfield{Aktor główny:}{Magazynier}
%   %
%   %	Warunki początkowe
%   \addfield{Warunki początkowe:}{Magazynier został uwierzytelniony.}
%   %
%   % Warunki końcowe (brak).
%   % 
%   % Scenariusz główny.
%   \addscenario{Przebieg zdarzeń:}{
%   \item Magazynier wybiera opcję wyświetlenia listy towarów w
%     magazynie.
%   \item System wyświetla listę towarów w magazynie.
%   }
%   %
%   % Wyjątki \addfield{Wyjątki:}{brak}
%   %
%   % Wymagania funkcjonalne związane z danym przypadkiem użycia
%   \addfield{Wymagania funkcjonalne:}{Usuwanie danych towarów}
%   %
%   % Wymagania niefunkcjonalne - tylko gdy konieczne
%   % \addfield{Wymagania niefunkcjonalne}{}
% \end{usecase}
% TODO diagram komunikacji TODO diagram ecb

% \subsection{Zarządzanie zamówieniami zakupu}
% \begin{usecase}
%   %
%   % ID + nazwa
%   \addtitle{PU36}{Zarządzanie zamówieniami zakupu}
%   %
%   % Priorytet
%   \addfield{Priorytet:}{wysoki}
%   %
%   % Aktor główny - zazwyczaj ten, który inicjuje przypadek użycia.
%   \addfield{Aktor główny:}{Magazynier}
%   %
%   %	Warunki początkowe
%   \addfield{Warunki początkowe:}{Magazynier został uwierzytelniony.}
%   %
%   % Warunki końcowe (brak).
%   % 
%   % Scenariusz główny.
%   \addscenario{Przebieg zdarzeń:}{
%   \item Magazynier wybiera opcję wyświetlenia listy towarów w
%     magazynie.
%   \item System wyświetla listę towarów w magazynie.
%   }
%   %
%   % Wyjątki \addfield{Wyjątki:}{brak}
%   %
%   % Wymagania funkcjonalne związane z danym przypadkiem użycia
%   \addfield{Wymagania funkcjonalne:}{Usuwanie danych towarów}
%   %
%   % Wymagania niefunkcjonalne - tylko gdy konieczne
%   % \addfield{Wymagania niefunkcjonalne}{}
% \end{usecase}
% TODO diagram komunikacji TODO diagram ecb

% \subsection{Zarządzanie zamówieniami zakupu}
% \begin{usecase}
%   %
%   % ID + nazwa
%   \addtitle{PU36}{Zarządzanie zamówieniami zakupu}
%   %
%   % Priorytet
%   \addfield{Priorytet:}{wysoki}
%   %
%   % Aktor główny - zazwyczaj ten, który inicjuje przypadek użycia.
%   \addfield{Aktor główny:}{Magazynier}
%   %
%   %	Warunki początkowe
%   \addfield{Warunki początkowe:}{Magazynier został uwierzytelniony.}
%   %
%   % Warunki końcowe (brak).
%   % 
%   % Scenariusz główny.
%   \addscenario{Przebieg zdarzeń:}{
%   \item Magazynier wybiera opcję wyświetlenia listy towarów w
%     magazynie.
%   \item System wyświetla listę towarów w magazynie.
%   }
%   %
%   % Wyjątki \addfield{Wyjątki:}{brak}
%   %
%   % Wymagania funkcjonalne związane z danym przypadkiem użycia
%   \addfield{Wymagania funkcjonalne:}{Usuwanie danych towarów}
%   %
%   % Wymagania niefunkcjonalne - tylko gdy konieczne
%   % \addfield{Wymagania niefunkcjonalne}{}
% \end{usecase}
% TODO diagram komunikacji TODO diagram ecb

% \subsection{Zarządzanie zamówieniami zakupu}
% \begin{usecase}
%   %
%   % ID + nazwa
%   \addtitle{PU36}{Zarządzanie zamówieniami zakupu}
%   %
%   % Priorytet
%   \addfield{Priorytet:}{wysoki}
%   %
%   % Aktor główny - zazwyczaj ten, który inicjuje przypadek użycia.
%   \addfield{Aktor główny:}{Magazynier}
%   %
%   %	Warunki początkowe
%   \addfield{Warunki początkowe:}{Magazynier został uwierzytelniony.}
%   %
%   % Warunki końcowe (brak).
%   % 
%   % Scenariusz główny.
%   \addscenario{Przebieg zdarzeń:}{
%   \item Magazynier wybiera opcję wyświetlenia listy towarów w
%     magazynie.
%   \item System wyświetla listę towarów w magazynie.
%   }
%   %
%   % Wyjątki \addfield{Wyjątki:}{brak}
%   %
%   % Wymagania funkcjonalne związane z danym przypadkiem użycia
%   \addfield{Wymagania funkcjonalne:}{Usuwanie danych towarów}
%   %
%   % Wymagania niefunkcjonalne - tylko gdy konieczne
%   % \addfield{Wymagania niefunkcjonalne}{}
% \end{usecase}
% TODO diagram komunikacji TODO diagram ecb






