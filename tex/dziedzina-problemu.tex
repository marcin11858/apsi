\section{Dziedzina problemu}
\label{dziedzina-problemu}
Poniżej przedstawione zostały kluczowe pojęcia i procesy
charakteryzujące system obsługi magazynu.

\subsection{Towary}
Towar jest podstawową jednostką przechowywaną w magazynie. Każdy towar
jest opisany przez następujące cechy:
\begin{itemize}
\item nazwa,
\item kod kreskowy,
\item kategoria,
\item jednostka miary,
\item stan początkowy (początkowa ilość towaru w magazynie),
\item cena sprzedaży.
\end{itemize}

Warunki poprawności danych towaru:
\begin{itemize}
\item nazwa towaru powinna posiadać co najmniej 3 znaki;
\item cena sprzedaży musi być liczbą nieujemną, z dokładnością do
  dwóch miejsc po przecinku;
\item stan początkowy musi być liczbą całkowitą nieujemną;
\item kod kreskowy musi być zgodny ze standardem \textbf{EAN}.
\end{itemize}

Możliwe są następujące operacje:
\begin{itemize}
\item dodawanie towaru,
\item edytowanie towaru z wyłączeniem bezpośredniej edycji stanu
  początkowego, który jest cechą ustalaną na podstawie dokumentów PZ,
  WZ oraz korekt stanu magazynu,
\item usuwanie towarów, które nie znajdują się w żadnym z dokumentów
  PZ, WZ lub korekcie magazynu.
\end{itemize}

Ponadto dla każdego towaru możliwe jest określenie stanu minimalnego
oraz maksymalnego w magazynie, od którego zależna będzie możliwość
dokonania operacji przyjęcia zamówienia, wydania zamówienia oraz
korekty stanu magazynu.

\subsection{Dokumenty}
W systemie wyróżnione są trzy typy dokumentów: PZ, WZ oraz korekta
stanu magazynu.

\subsubsection{PZ}
Dokument PZ jest dokumentem powiązanym z procesem przyjęcia towaru.
Dokument PZ zawiera:
\begin{itemize}
\item numer dokumentu: PZnr\_kolejny\_dokumentu/rok, np: PZ1/2003,
\item dane wystawcy (magazynu),
\item dane kontrahenta,
\item datę operacji,
\item dane towarów wraz z ilością przyjętych towarów -- wartość ta
  dodaje się do stanu towaru w magazynie,
\item cenę towaru, ustalaną indywidualnie dla każdego dokumentu PZ,
\item łączną cenę towarów.
\end{itemize}

\subsubsection{WZ}
Dokument WZ jest dokumentem powiązanym z procesem wydania towaru.
Dokument WZ zawiera:
\begin{itemize}
\item numer dokumentu: WZnr\_kolejny\_dokumentu/rok, np: WZ1/2003,
\item dane wystawcy (magazynu),
\item dane kontrahenta,
\item datę operacji,
\item dane towarów wraz z ilością przyjętych towarów -- wartość ta
  odejmuje się do stanu towaru w magazynie,
\item cenę towaru, ustalaną indywidualnie dla każdego dokumentu WZ
  (cena towaru z danych towaru to tylko sugestia),
\item łączną cenę towarów.
\end{itemize}

\subsubsection{Korekta stanu magazynu}
Dokument korekty stanu magazynu jest wypełniany w przypadku
konieczności aktualizacji ilości faktycznie przechowywanych towarów.
Dokument ten może być użyty np. w przypadku upływu terminu ważności
przechowywanych towarów lub ich kradzieży.
\begin{itemize}
\item data korekty,
\item numer dokumentu,
\item powód korekty,
\item dane towarów, których stan podlega korekcie,
\item nowy stan towaru,
\item suma modyfikowanej ilości towarów na dokumencie.
\end{itemize}

Warunki poprawności danych korekty stanu magazynu:
\begin{itemize}
\item data korekty musi spełniać format określony przez dokument \emph{ISO 8601:2004},
\item numer dokumentu musi posiadać format Knr\_kolejny\_dokumentu/rok, np. K1/2013,
\item powód korekty jest polem wymaganym,
\item dokument musi zawierać informację o co najmniej jednym towarze,
\item nowy stan towaru musi być liczbą nieujemną.
\end{itemize}

\subsubsection{Korekta PZ}
Korekta PZ formalizuje operację zwrotu towaru do dostawcy
(kontrahenta).  Korekta PZ jest dokumentem analogicznym do dokumentu
PZ, z zastrzeżeniem, iż odnosi się ona do już istniejącego dokumentu
PZ oraz dane z korekty PZ muszą być zgodne z danymi dokumentu PZ do
którego ta korekta się odnosi -- w korekcie może znaleźć się tylko
podzbiór towarów z dokumentu PZ a ilości poszczególnych towarów na
korekcie nie mogą przekraczać ilości towarów oryginalnego dokumentu.
% Przydałaby się pewnie też możliwość realizacji korekty do korekty PZ
% itd.  ale w ramach tego projektu odpuszczamy sobie to ;P.

Ponadto:
\begin{itemize}
\item numer dokumentu: PZknr\_kolejny\_dokumentu/rok, np PZk1/2013
  (numer kolejny dokumentu nie jest w żaden sposób powiązany z
  jakimkolwiek numerem istniejącego dokumentu),
\item cena towaru w korekcie PZ ustalana jest niezależnie od ceny z
  oryginalnego dokumentu PZ,
\item ilość towaru w korekcie PZ odejmuje się od stanu towaru w
  magazynie.
\end{itemize}

\subsubsection{Korekta WZ}
Korekta WZ formalizuje operację zwrotu towaru od odbiorcy
(kontrahenta) Korekta WZ jest dokumentem analogicznym do dokumentu WZ,
z zastrzeżeniem, iż odnosi się ona do już istniejącego dokumentu WZ
oraz dane z korekty WZ muszą być zgodne z danymi dokumentu WZ do
którego ta korekta się odnosi -- w korekcie może znaleźć się tylko
podzbiór towarów z dokumentu WZ a ilości poszczególnych towarów na
korekcie nie mogą przekraczać ilości towarów oryginalnego dokumentu.
% Przydałaby się pewnie też możliwość realizacji korekty do korekty WZ
% itd.  ale w ramach tego projektu odpuszczamy sobie to ;P.

Ponadto:
\begin{itemize}
\item numer dokumentu: WZknr\_kolejny\_dokumentu/rok, np WZk1/2013
  (numer kolejny dokumentu nie jest w żaden sposób powiązany z
  jakimkolwiek numerem istniejącego dokumentu),
\item cena towaru w korekcie WZ ustalana jest niezależnie od ceny z
  oryginalnego dokumentu WZ,
\item ilość towaru w korekcie WZ dodaje się od stanu towaru w
  magazynie.
\end{itemize}

\subsection{Przyjęcie towaru}
Z procesem przyjęcia towaru powiązane są następujące operacje:
\begin{itemize}
\item utworzenie dokumentu PZ,
\item edycja dokumentu PZ -- dowolne pole, zmiany ilości przyjętego
  towaru wpływają na stan towaru w magazynie,
\item usunięcie dokumentu PZ -- usunięcie dokumentu PZ powoduje
  aktualizację (odjęcie) ilości towarów, które były uwzględnione w tym
  dokumencie.
\end{itemize}

\subsection{Wydanie towaru}
Z procesem wydania towaru powiązane są następujące operacje:
\begin{itemize}
\item utworzenie dokumentu WZ,
\item edycja dokumentu WZ -- dowolne pole, zmiany ilości przyjętego
  towaru wpływają na stan towaru w magazynie,
\item usunięcie dokumentu WZ -- usunięcie dokumentu WZ powoduje
  aktualizację (dodanie) ilości towarów, które były uwzględnione w tym
  dokumencie.
\end{itemize}

Dodatkowo można wskazać przyjęcie towaru, do którego odnosi się to
wydanie towaru: wtedy możliwe jest wskazanie wydawanego towaru, który
był kiedyś przyjęty za pewną kwotę, a tym samym możliwe jest
obliczenie zysku z wydania towaru/wskazanie marży z towaru).

% LocalWords: WZ
