\section{Moduł zarządzania systemem}
\singlespacing

\subsection{Zarządzanie pracownikami} 

\begin{usecase}
% Pola na podstawie przykładowego szablonu z http://piotrsalata.pl/documents/wymagania.pdf.
% ID + nazwa
\addtitle{PU1}{Zarządzanie pracownikami} 
% Priorytet
\addfield{Priorytet:}{wysoki}
% Aktor główny - zazwyczaj ten, który inicjuje przypadek użycia.
\addfield{Aktor główny:}{Kierownik}
% Pozostali aktorzy uczestniczący 
%\additemizedfield{Aktorzy:}{
%	\item Sprzedawca
%	\item Magazynier
%}
% Warunki początkowe
\additemizedfield{Warunki początkowe:}{
  \item Kierownik został uwierzytelniony.
} 
% Warunki końcowe
%\addfield{Warunki końcowe:} 
% Scenariusz główny
\addscenario{Przebieg zdarzeń:}{
	\item Kierownik wybiera opcję przeglądania pracowników.
    \item System wyświetla listę pracowników.
}
% Wyjątki
% \addfield{Wyjątki}{} TODO uwzględniać to?
% Wymagania funkcjonalne związane z danym przypadkiem użycia
\addfield{Wymagania funkcjonalne}{1. Zarządzanie pracownikami}
% Wymagania niefunkcjonalne - tylko gdy konieczne
% \addfield{Wymagania niefunkcjonalne}{}
\end{usecase}

\subsection{Dodawanie pracowników}
\begin{usecase}
% Pola na podstawie przykładowego szablonu z http://piotrsalata.pl/documents/wymagania.pdf.
% ID + nazwa
\addtitle{PU2}{Dodawanie pracowników} 
% Priorytet
\addfield{Priorytet:}{wysoki}
% Aktor główny - zazwyczaj ten, który inicjuje przypadek użycia.
\addfield{Aktor główny:}{Kierownik}
% Pozostali aktorzy uczestniczący 
%\additemizedfield{Aktorzy:}{
%	\item Sprzedawca
%	\item Magazynier
%}
% Warunki początkowe
\additemizedfield{Warunki początkowe:}{
  \item Kierownik został uwierzytelniony.
} 
% Warunki końcowe
\addfield{Warunki końcowe:}{Dane pracownika zostają zapisane w systemie}
% Scenariusz główny
\addscenario{Przebieg zdarzeń:}{
	\item Kierownik wybiera opcję dodania nowego pracownika.
    \item System wyświetla formularz umożliwiający wprowadzenie danych pracownika.
    \item Kierownik wprowadza dane pracownika.
    \item Kierownik wciska przycisk "Zapisz"
    \item System sprawdza poprawność wprowadzonych danych.
    \item System zapisuje dane nowego pracownika.
    \item System wyświetla potwierdzenie wykonania operacji.
}
% Scenariusze alternatywne
\addscenario{Przebiegi alternatywne:}{
  % [nr kroku scenariusza głównego.kolejna litera a,b,c wyznaczająca kolejny scenariusz alternatywny od wybranego kroku s.głównego.
	\item[6.a] Wprowadzono błędne dane
		\begin{enumerate}
		\item[1.--5.] Jak w scenariuszu głównym.
		\item[6.] System wyświetla komunikat informujący o wprowadzeniu błędnych danych.
		\end{enumerate}
}
% Wyjątki
% \addfield{Wyjątki}{} TODO uwzględniać to?
% Wymagania funkcjonalne związane z danym przypadkiem użycia
\addfield{Wymagania funkcjonalne}{1. Zarządzanie pracownikami}
% Wymagania niefunkcjonalne - tylko gdy konieczne
% \addfield{Wymagania niefunkcjonalne}{}
\end{usecase}

\subsection{Edycja danych pracowników}
\begin{usecase}
% Pola na podstawie przykładowego szablonu z http://piotrsalata.pl/documents/wymagania.pdf.
% ID + nazwa
\addtitle{PU3}{Edycja danych pracowników} 
% Priorytet
\addfield{Priorytet:}{wysoki}
% Aktor główny - zazwyczaj ten, który inicjuje przypadek użycia.
\addfield{Aktor główny:}{Kierownik}
% Pozostali aktorzy uczestniczący 
%\additemizedfield{Aktorzy:}{
%	\item Sprzedawca
%	\item Magazynier
%}
% Warunki początkowe
\additemizedfield{Warunki początkowe:}{
  \item Kierownik został uwierzytelniony.
  \item W systemie istnieje co najmniej jeden pracownik.
} 
% Warunki końcowe
\addfield{Warunki końcowe:}{Dane pracownika zostają zmienione w systemie}
% Scenariusz główny
\addscenario{Przebieg zdarzeń:}{
	\item Kierownik wybiera opcję edycji pracownika.
    \item System wyświetla formularz umożliwiający modyfikację danych pracownika.
    \item Kierownik modyfikuje dane pracownika.
    \item Kierownik wciska przycisk "Zapisz"
    \item System sprawdza poprawność wprowadzonych danych.
    \item System zapisuje zmienione dane pracownika.
    \item System wyświetla potwierdzenie wykonania operacji.
}
% Scenariusze alternatywne
\addscenario{Przebiegi alternatywne:}{
  % [nr kroku scenariusza głównego.kolejna litera a,b,c wyznaczająca kolejny scenariusz alternatywny od wybranego kroku s.głównego.
	\item[6.a] Wprowadzono błędne dane
		\begin{enumerate}
		\item[1.--5.] Jak w scenariuszu głównym.
		\item[6.] System wyświetla komunikat informujący o wprowadzeniu błędnych danych.
		\end{enumerate}
}
% Wyjątki
% \addfield{Wyjątki}{} TODO uwzględniać to?
% Wymagania funkcjonalne związane z danym przypadkiem użycia
\addfield{Wymagania funkcjonalne}{1. Zarządzanie pracownikami}
% Wymagania niefunkcjonalne - tylko gdy konieczne
% \addfield{Wymagania niefunkcjonalne}{}
\end{usecase}

\subsection{Usuwanie danych pracowników}
\begin{usecase}
% Pola na podstawie przykładowego szablonu z http://piotrsalata.pl/documents/wymagania.pdf.
% ID + nazwa
\addtitle{PU4}{Usuwanie danych pracowników} 
% Priorytet
\addfield{Priorytet:}{wysoki}
% Aktor główny - zazwyczaj ten, który inicjuje przypadek użycia.
\addfield{Aktor główny:}{Kierownik}
% Pozostali aktorzy uczestniczący 
%\additemizedfield{Aktorzy:}{
%	\item Sprzedawca
%	\item Magazynier
%}
% Warunki początkowe
\additemizedfield{Warunki początkowe:}{
  \item Kierownik został uwierzytelniony.
  \item W istnieje co najmniej jeden pracownik.
} 
% Warunki końcowe
\addfield{Warunki końcowe:}{Dane pracownika zostają usunięte z systemu.}
% Scenariusz główny
\addscenario{Przebieg zdarzeń:}{
	\item Kierownik wybiera opcję usunięcia pracownika.
    \item System wyświetla okno potwierdzenia.
    \item Kierownik zatwierdza usunięcie pracownika.
    \item System usuwa dane pracownika.
    \item System wyświetla potwierdzenie wykonania operacji.
}
% Scenariusze alternatywne
\addscenario{Przebiegi alternatywne:}{
  % [nr kroku scenariusza głównego.kolejna litera a,b,c wyznaczająca kolejny scenariusz alternatywny od wybranego kroku s.głównego.
	\item[3.a] Kierownik anuluje usuwanie pracownika
		\begin{enumerate}
		\item[1.--2.] Jak w scenariuszu głównym.
		\item[3.] System zamyka okno potwierdzenia.
		\end{enumerate}
}
% Wyjątki
% \addfield{Wyjątki}{} TODO uwzględniać to?
% Wymagania funkcjonalne związane z danym przypadkiem użycia
\addfield{Wymagania funkcjonalne}{1. Zarządzanie pracownikami}
% Wymagania niefunkcjonalne - tylko gdy konieczne
% \addfield{Wymagania niefunkcjonalne}{}
\end{usecase}