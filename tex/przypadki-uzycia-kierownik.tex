\section{Moduł zarządzania systemem}
\singlespacing

\subsection{Zarządzanie pracownikami} 
\begin{usecase}
\addtitle{PU1}{Zarządzanie pracownikami} 
\addfield{Priorytet:}{wysoki}
\addfield{Aktor główny:}{Kierownik}
\addfield{Rozszerza przypadki:}{PU1}
\additemizedfield{Warunki początkowe:}{
  \item Aktor został uwierzytelniony.
} 
\addscenario{Scenariusz główny:}{
	\item Aktor wybiera opcję przeglądania pracowników.
    \item System wyświetla listę pracowników.
}
\addfield{Wymagania funkcjonalne}{1. Zarządzanie pracownikami}
\end{usecase}

\subsection{Dodawanie pracowników}
\begin{usecase}
\addtitle{PU2}{Dodawanie pracowników} 
% Priorytet
\addfield{Priorytet:}{wysoki}
\addfield{Aktor główny:}{Kierownik}
\addfield{Rozszerza przypadki:}{PU1}
\additemizedfield{Warunki początkowe:}{
  \item Aktor został uwierzytelniony.
} 
\addfield{Warunki końcowe:}{Dane pracownika zostają zapisane w systemie}
\addscenario{Scenariusz główny:}{
	\item Aktor wybiera opcję dodania nowego pracownika.
    \item System wyświetla formularz umożliwiający wprowadzenie danych pracownika.
    \item Aktor wprowadza dane pracownika.
    \item Aktor zatwierdza wprowadzone dane.
    \item System sprawdza poprawność wprowadzonych danych.
    \item System zapisuje dane nowego pracownika.
    \item System wyświetla potwierdzenie wykonania operacji.
}
\addscenario{Scenariusz alternatywny:}{
	\item[6.a] Wprowadzono błędne dane
		\begin{enumerate}
		\item[1.--5.] Jak w scenariuszu głównym.
		\item[6.] System wyświetla komunikat informujący o wprowadzeniu błędnych danych.
		\item[7.] Powrót do punktu 3. scenariusza głównego.
		\end{enumerate}
}
\addfield{Wymagania funkcjonalne}{1. Zarządzanie pracownikami, 1.1 Dodawanie pracowników}
\end{usecase}

\subsection{Edycja danych pracowników}
\begin{usecase}
\addtitle{PU3}{Edycja danych pracowników} 
\addfield{Priorytet:}{wysoki}
\addfield{Aktor główny:}{Kierownik}
\addfield{Rozszerza przypadki:}{PU1}
\additemizedfield{Warunki początkowe:}{
  \item Aktor został uwierzytelniony.
  \item W systemie istnieje co najmniej jeden pracownik.
} 
\addfield{Warunki końcowe:}{Dane pracownika zostają zmienione w systemie}
\addscenario{Scenariusz główny:}{
	\item Aktor wybiera opcję edycji pracownika.
    \item System wyświetla formularz umożliwiający modyfikację danych pracownika.
    \item Aktor modyfikuje dane pracownika.
    \item Aktor zatwierdza wprowadzone zmiany.
    \item System sprawdza poprawność wprowadzonych danych.
    \item System zapisuje zmienione dane pracownika.
    \item System wyświetla potwierdzenie wykonania operacji.
}
\addscenario{Scenariusz alternatywny:}{
	\item[6.a] Wprowadzono błędne dane
		\begin{enumerate}
		\item[1.--5.] Jak w scenariuszu głównym.
		\item[6.] System wyświetla komunikat informujący o wprowadzeniu błędnych danych.
		\item[7.] Powrót do punktu 3. scenariusza głównego.
		\end{enumerate}
}
\addfield{Wymagania funkcjonalne}{1. Zarządzanie pracownikami, 1.2 Edycja danych pracowników}
\end{usecase}

\subsection{Usuwanie danych pracowników}
\begin{usecase}
\addtitle{PU4}{Usuwanie danych pracowników} 
\addfield{Priorytet:}{wysoki}
\addfield{Aktor główny:}{Kierownik}
\addfield{Rozszerza przypadki:}{PU1}
\additemizedfield{Warunki początkowe:}{
  \item Aktor został uwierzytelniony.
  \item W istnieje co najmniej jeden pracownik.
} 
\addfield{Warunki końcowe:}{Dane pracownika zostają usunięte z systemu.}
\addscenario{Scenariusz główny:}{
	\item Aktor wybiera opcję usunięcia pracownika.
    \item System wyświetla okno potwierdzenia.
    \item Aktor zatwierdza usunięcie pracownika.
    \item System usuwa dane pracownika.
    \item System wyświetla potwierdzenie wykonania operacji.
}
\addscenario{Scenariusz alternatywny:}{
	\item[3.a] Aktor anuluje usuwanie pracownika
		\begin{enumerate}
		\item[1.--2.] Jak w scenariuszu głównym.
		\item[3.] System zamyka okno potwierdzenia.
		\end{enumerate}
}
\addfield{Wymagania funkcjonalne}{1. Zarządzanie pracownikami, 1.4 Usuwanie danych pracowników}
\end{usecase}

\subsection{Zarządzanie raportami} 
\begin{usecase}
\addtitle{PU5}{Zarządzanie raportami} 
\addfield{Priorytet:}{średni}
\addfield{Aktor główny:}{Kierownik}
\additemizedfield{Warunki początkowe:}{
  \item Aktor został uwierzytelniony.
  \item W systemie przechowywana jest informacja o co najmniej jednym przyjęciu/wydaniu towarów.
} 
\addscenario{Scenariusz główny:}{
	\item Aktor wybiera opcję tworzenia raportów.
    \item System wyświetla listę możliwych dostępnych typów raportów.
}
\addfield{Wymagania funkcjonalne}{7. Zarządzanie raportami}
\end{usecase}

\subsection{Generowanie raportów o wydanych towarach}
\begin{usecase}
\addtitle{PU6}{Generowanie raportów o wydanych towarach} 
\addfield{Priorytet:}{średni}
\addfield{Aktor główny:}{Kierownik}
\addfield{Rozszerza przypadki:}{PU5}
\additemizedfield{Warunki początkowe:}{
  \item Aktor został uwierzytelniony.
  \item W systemie przechowywana jest informacja o co najmniej jednym wydaniu towarów.
} 
\addfield{Warunki końcowe:} {Aktor posiada dokument z raportem o wydanych towarach.} 
\addscenario{Scenariusz główny:}{
	\item Aktor wybiera opcję generowania raportu o wydanych towarach.
    \item System generuje raport o wydanych towarach.
}
\addfield{Wymagania funkcjonalne}{7. Zarządzanie raportami, 7.1 Generowanie raportów o wydanych towarach}
\end{usecase}

\subsection{Generowanie raportów o przyjętych towarach}
\begin{usecase}
\addtitle{PU7}{Generowanie raportów o przyjętych towarach} 
\addfield{Priorytet:}{średni}
\addfield{Aktor główny:}{Kierownik}
\addfield{Rozszerza przypadki:}{PU5}
\additemizedfield{Warunki początkowe:}{
  \item Aktor został uwierzytelniony.
  \item W systemie przechowywana jest informacja o co najmniej jednym przyjęciu towarów.
} 
\addfield{Warunki końcowe:} {Aktor posiada dokument z raportem o przyjętych towarach.} 
\addscenario{Scenariusz główny:}{
	\item Aktor wybiera opcję generowania raportu o przyjętych towarach.
    \item System generuje raport o przyjętych towarach.
}
\addfield{Wymagania funkcjonalne}{7. Zarządzanie raportami, 7.2 Generowanie raportów o przyjętych towarach}
\end{usecase}