\section{Moduł zarządzania systemem}
\singlespacing

\subsection{Zarządzanie pracownikami} 
Przypadek użycia umożliwiający zarządzanie pracownikami.
\begin{usecase}
% Pola na podstawie przykładowego szablonu z http://piotrsalata.pl/documents/wymagania.pdf.
% ID + nazwa
\addtitle{PU1}{Zarządzanie pracownikami} 
% Priorytet
\addfield{Priorytet:}{wysoki}
% Aktor główny - zazwyczaj ten, który inicjuje przypadek użycia.
\addfield{Aktor główny:}{Kierownik}
% Pozostali aktorzy uczestniczący 
%\additemizedfield{Aktorzy:}{
%	\item Sprzedawca
%	\item Magazynier
%}
% Warunki początkowe
\additemizedfield{Warunki początkowe:}{
  \item Aktor został uwierzytelniony.
} 
% Warunki końcowe
%\addfield{Warunki końcowe:} 
% Scenariusz główny
\addscenario{Scenariusz główny:}{
	\item Aktor wybiera opcję przeglądania pracowników.
    \item System wyświetla listę pracowników.
}
% Wyjątki
% \addfield{Wyjątki}{} TODO uwzględniać to?
% Wymagania funkcjonalne związane z danym przypadkiem użycia
\addfield{Wymagania funkcjonalne}{1. Zarządzanie pracownikami}
% Wymagania niefunkcjonalne - tylko gdy konieczne
% \addfield{Wymagania niefunkcjonalne}{}
\end{usecase}

\subsection{Dodawanie pracowników}
Przypadek użycia umożliwiający dodawanie pracowników.
\begin{usecase}
% Pola na podstawie przykładowego szablonu z http://piotrsalata.pl/documents/wymagania.pdf.
% ID + nazwa
\addtitle{PU2}{Dodawanie pracowników} 
% Priorytet
\addfield{Priorytet:}{wysoki}
% Aktor główny - zazwyczaj ten, który inicjuje przypadek użycia.
\addfield{Aktor główny:}{Kierownik}
% Pozostali aktorzy uczestniczący 
%\additemizedfield{Aktorzy:}{
%	\item Sprzedawca
%	\item Magazynier
%}
% Warunki początkowe
\additemizedfield{Warunki początkowe:}{
  \item Aktor został uwierzytelniony.
} 
% Warunki końcowe
\addfield{Warunki końcowe:}{Dane pracownika zostają zapisane w systemie}
% Scenariusz główny
\addscenario{Scenariusz główny:}{
	\item Aktor wybiera opcję dodania nowego pracownika.
    \item System wyświetla formularz umożliwiający wprowadzenie danych pracownika.
    \item Aktor wprowadza dane pracownika.
    \item Aktor zatwierdza wprowadzone dane.
    \item System sprawdza poprawność wprowadzonych danych.
    \item System zapisuje dane nowego pracownika.
    \item System wyświetla potwierdzenie wykonania operacji.
}
% Scenariusze alternatywne
\addscenario{Scenariusz alternatywny:}{
  % [nr kroku scenariusza głównego.kolejna litera a,b,c wyznaczająca kolejny scenariusz alternatywny od wybranego kroku s.głównego.
	\item[6.a] Wprowadzono błędne dane
		\begin{enumerate}
		\item[1.--5.] Jak w scenariuszu głównym.
		\item[6.] System wyświetla komunikat informujący o wprowadzeniu błędnych danych.
		\item[7.] Powrót do punktu 3. scenariusza głównego.
		\end{enumerate}
}
% Wyjątki
% \addfield{Wyjątki}{} TODO uwzględniać to?
% Wymagania funkcjonalne związane z danym przypadkiem użycia
\addfield{Wymagania funkcjonalne}{1. Zarządzanie pracownikami}
% Wymagania niefunkcjonalne - tylko gdy konieczne
% \addfield{Wymagania niefunkcjonalne}{}
\end{usecase}

\subsection{Edycja danych pracowników}
Przypadek użycia opisujący edycję danych pracowników.
\begin{usecase}
% Pola na podstawie przykładowego szablonu z http://piotrsalata.pl/documents/wymagania.pdf.
% ID + nazwa
\addtitle{PU3}{Edycja danych pracowników} 
% Priorytet
\addfield{Priorytet:}{wysoki}
% Aktor główny - zazwyczaj ten, który inicjuje przypadek użycia.
\addfield{Aktor główny:}{Kierownik}
% Pozostali aktorzy uczestniczący 
%\additemizedfield{Aktorzy:}{
%	\item Sprzedawca
%	\item Magazynier
%}
% Warunki początkowe
\additemizedfield{Warunki początkowe:}{
  \item Aktor został uwierzytelniony.
  \item W systemie istnieje co najmniej jeden pracownik.
} 
% Warunki końcowe
\addfield{Warunki końcowe:}{Dane pracownika zostają zmienione w systemie}
% Scenariusz główny
\addscenario{Scenariusz główny:}{
	\item Aktor wybiera opcję edycji pracownika.
    \item System wyświetla formularz umożliwiający modyfikację danych pracownika.
    \item Aktor modyfikuje dane pracownika.
    \item Aktor zatwierdza wprowadzone zmiany.
    \item System sprawdza poprawność wprowadzonych danych.
    \item System zapisuje zmienione dane pracownika.
    \item System wyświetla potwierdzenie wykonania operacji.
}
% Scenariusze alternatywne
\addscenario{Scenariusz alternatywny:}{
  % [nr kroku scenariusza głównego.kolejna litera a,b,c wyznaczająca kolejny scenariusz alternatywny od wybranego kroku s.głównego.
	\item[6.a] Wprowadzono błędne dane
		\begin{enumerate}
		\item[1.--5.] Jak w scenariuszu głównym.
		\item[6.] System wyświetla komunikat informujący o wprowadzeniu błędnych danych.
		\item[7.] Powrót do punktu 3. scenariusza głównego.
		\end{enumerate}
}
% Wyjątki
% \addfield{Wyjątki}{} TODO uwzględniać to?
% Wymagania funkcjonalne związane z danym przypadkiem użycia
\addfield{Wymagania funkcjonalne}{1. Zarządzanie pracownikami}
% Wymagania niefunkcjonalne - tylko gdy konieczne
% \addfield{Wymagania niefunkcjonalne}{}
\end{usecase}

\subsection{Usuwanie danych pracowników}
Przypadek użycia opisujący usuwanie danych pracowników.
\begin{usecase}
% Pola na podstawie przykładowego szablonu z http://piotrsalata.pl/documents/wymagania.pdf.
% ID + nazwa
\addtitle{PU4}{Usuwanie danych pracowników} 
% Priorytet
\addfield{Priorytet:}{wysoki}
% Aktor główny - zazwyczaj ten, który inicjuje przypadek użycia.
\addfield{Aktor główny:}{Kierownik}
% Pozostali aktorzy uczestniczący 
%\additemizedfield{Aktorzy:}{
%	\item Sprzedawca
%	\item Magazynier
%}
% Warunki początkowe
\additemizedfield{Warunki początkowe:}{
  \item Aktor został uwierzytelniony.
  \item W istnieje co najmniej jeden pracownik.
} 
% Warunki końcowe
\addfield{Warunki końcowe:}{Dane pracownika zostają usunięte z systemu.}
% Scenariusz główny
\addscenario{Scenariusz główny:}{
	\item Aktor wybiera opcję usunięcia pracownika.
    \item System wyświetla okno potwierdzenia.
    \item Aktor zatwierdza usunięcie pracownika.
    \item System usuwa dane pracownika.
    \item System wyświetla potwierdzenie wykonania operacji.
}
% Scenariusze alternatywne
\addscenario{Scenariusz alternatywny:}{
  % [nr kroku scenariusza głównego.kolejna litera a,b,c wyznaczająca kolejny scenariusz alternatywny od wybranego kroku s.głównego.
	\item[3.a] Aktor anuluje usuwanie pracownika
		\begin{enumerate}
		\item[1.--2.] Jak w scenariuszu głównym.
		\item[3.] System zamyka okno potwierdzenia.
		\end{enumerate}
}
% Wyjątki
% \addfield{Wyjątki}{} TODO uwzględniać to?
% Wymagania funkcjonalne związane z danym przypadkiem użycia
\addfield{Wymagania funkcjonalne}{1. Zarządzanie pracownikami}
% Wymagania niefunkcjonalne - tylko gdy konieczne
% \addfield{Wymagania niefunkcjonalne}{}
\end{usecase}

\subsection{Zarządzanie raportami} 
Przypadek użycia umożliwiający zarządzanie raportami.
\begin{usecase}
% Pola na podstawie przykładowego szablonu z http://piotrsalata.pl/documents/wymagania.pdf.
% ID + nazwa
\addtitle{PU5}{Zarządzanie raportami} 
% Priorytet
\addfield{Priorytet:}{średni}
% Aktor główny - zazwyczaj ten, który inicjuje przypadek użycia.
\addfield{Aktor główny:}{Kierownik}
% Pozostali aktorzy uczestniczący 
%\additemizedfield{Aktorzy:}{
%	\item Sprzedawca
%	\item Magazynier
%}
% Warunki początkowe
\additemizedfield{Warunki początkowe:}{
  \item Aktor został uwierzytelniony.
  \item W systemie przechowywana jest informacja o co najmniej jednej sprzedaży/zakupie.
} 
% Warunki końcowe
%\addfield{Warunki końcowe: } 
% Scenariusz główny
\addscenario{Scenariusz główny:}{
	\item Aktor wybiera opcję tworzenia raportów.
    \item System wyświetla listę możliwych dostępnych typów raportów.
}
% Wyjątki
% \addfield{Wyjątki}{} TODO uwzględniać to?
% Wymagania funkcjonalne związane z danym przypadkiem użycia
\addfield{Wymagania funkcjonalne}{7. Zarządzanie raportami}
% Wymagania niefunkcjonalne - tylko gdy konieczne
% \addfield{Wymagania niefunkcjonalne}{}
\end{usecase}

\subsection{Generowanie raportów sprzedaży}
Przypadek użycia umożliwiający generowanie raportów sprzedaży.
\begin{usecase}
% Pola na podstawie przykładowego szablonu z http://piotrsalata.pl/documents/wymagania.pdf.
% ID + nazwa
\addtitle{PU6}{Generowanie raportów sprzedaży} 
% Priorytet
\addfield{Priorytet:}{średni}
% Aktor główny - zazwyczaj ten, który inicjuje przypadek użycia.
\addfield{Aktor główny:}{Kierownik}
% Pozostali aktorzy uczestniczący 
%\additemizedfield{Aktorzy:}{
%	\item Sprzedawca
%	\item Magazynier
%}
% Warunki początkowe
\additemizedfield{Warunki początkowe:}{
  \item Aktor został uwierzytelniony.
  \item W systemie przechowywana jest informacja o co najmniej jednej sprzedaży.
} 
% Warunki końcowe
\addfield{Warunki końcowe:} {Aktor posiada dokument z raportem sprzedaży.} 
% Scenariusz główny
\addscenario{Scenariusz główny:}{
	\item Aktor wybiera opcję dodania generowania raportu sprzedaży.
    \item System generuje raport sprzedaży.
}
% Scenariusze alternatywne
%\addscenario{Przebiegi alternatywne:}{
  % [nr kroku scenariusza głównego.kolejna litera a,b,c wyznaczająca kolejny scenariusz alternatywny od wybranego kroku s.głównego.
%}
% Wyjątki
% \addfield{Wyjątki}{} TODO uwzględniać to?
% Wymagania funkcjonalne związane z danym przypadkiem użycia
\addfield{Wymagania funkcjonalne}{7. Zarządzanie raportami}
% Wymagania niefunkcjonalne - tylko gdy konieczne
% \addfield{Wymagania niefunkcjonalne}{}
\end{usecase}