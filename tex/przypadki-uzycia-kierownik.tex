\section{Moduł zarządzania systemem}
\singlespacing
\begin{usecase}
% Pola na podstawie przykładowego szablonu z http://piotrsalata.pl/documents/wymagania.pdf.
% ID + nazwa
\addtitle{PU1}{Generowanie raportów sprzedaży} 
% Priorytet
\addfield{Priorytet:}{średni}
% Aktor główny - zazwyczaj ten, który inicjuje przypadek użycia.
\addfield{Aktor główny:}{Kierownik}
% Pozostali aktorzy uczestniczący 
%\additemizedfield{Aktorzy:}{
%	\item Sprzedawca
%	\item Magazynier
%}
% Warunki początkowe
\additemizedfield{Warunki początkowe:}{
  \item Kierownik został uwierzytelniony.
  \item W systemie przechowywana jest informacja o co najmniej jednej sprzedaży/zakupie.
} 
% Warunki końcowe
\addfield{Warunki końcowe:}{Kierownik posiada dokument z raportem z wybranej działalności hurtowni.}
% Scenariusz główny
\addscenario{Przebieg zdarzeń:}{
	\item Kierownik wybiera opcję tworzenia raportów.
    \item System wyświetla listę możliwych dostępnych typów raportów.
    \item 
}
% Scenariusze alternatywne
\addscenario{Przebiegi alternatywne:}{
  % [nr kroku scenariusza głównego.kolejna litera a,b,c wyznaczająca kolejny scenariusz alternatywny od wybranego kroku s.głównego.
	\item[2.a] Nie istnieją dane odpowiednie dla danego typu raportu. 
		\begin{enumerate}
		\item[1.] etc.
		\item[2.] etc.
		\end{enumerate}
}
% Wyjątki
% \addfield{Wyjątki}{} TODO uwzględniać to?
% Wymagania funkcjonalne związane z danym przypadkiem użycia
\addfield{Wymagania funkcjonalne}{7. Zarządzanie raportami}
% Wymagania niefunkcjonalne - tylko gdy konieczne
% \addfield{Wymagania niefunkcjonalne}{}
\end{usecase}
